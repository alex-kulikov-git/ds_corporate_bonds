\section{The Cashflow Use Case} \label{use-case}
In this chapter, we introduce the motivation behind our Cashflow use case, as well as the implementation approach of the corresponding prototype. 

\subsection{Motivation and Introduction of Use Case} \label{motivation}
In today's world, countless large construction projects are designed, planned and executed every year. Carrying out such projects often involves very high organizational effort from the organizing entity and a variety of contractors who work on the construction site. In the general case, a construction company sets up a legal agreement with a contractor to capture the details of the business deal. 
In this work, we want to propose an alternative system to the legal agreements on paper to guarantee temporal and financial efficiency. Based on Blockchain technology, the proposed model can be utilized to trace private or public legal agreements between multiple parties. This way, it is possible to keep track of the money flow, and of further important characteristics, such as the current state of the project, or its budget. Being a redundant system to actual paper contracts, we believe that this solution could help to prevent the embezzlement of project funds, intentional delays of work or even corruption due to the increased degree of transparency. \\ \\
Hyperledger Fabric allows the integration of different user roles with diverse user rights and as shown in Figure \ref{fig:use-case}, we have incorporated three user roles in our project that occur in every building project.
Starting from the top, an \textit{Authority} - e.g. a state - generally commissions the construction of the building and provides the necessary budget. Within the Blockchain network, the authority is analogous to the ''eyes'' of the building project, since it only has read access and can oversee the contracts, but not interfere with the legal agreements directly. 
In contrast, the \textit{Organizer} - e.g. a construction company - corresponds to the ''brains'' behind the construction which is responsible for i.a. querying, setting up, and signing legal agreements between itself and a given contractor. The organizer is provided with a budget by the authority and spends these funds on services. Further, this entity has all the necessary read and write permissions to submit transactions to the Blockchain. 
In our senses analogy, \textit{Contractors} - e.g. a painter or architect - are the ''hands'' of the project and provide their services to advance the construction site. As a user role, they have access rights which are similar to the Organizer with the only difference that they are only able to get legal agreements in which they are involved themselves. In the proposed network, contractors have the least amount of access rights and are the only entity which can exist multiple times.

\begin{figure}[]
	\centering
	\includegraphics[width=1\linewidth]{use-case.png}
	\caption{Visualization of Cashflow Use Case}
	\label{fig:use-case}
	\Description[Visualization of Cashflow Use Case]{Visualization of Cashflow Use Case}
	\vspace{-3mm} % reduce space after figure
\end{figure}
As a paradigm of a very large and lengthy construction site in Germany, we have focused on the new airport of Berlin within the implementation of our prototype.
When the construction began in 2006, the costs until opening in October 2011 were estimated to be 2 Billion Euros. Due to multiple scandals and several delays, the infrastructural building project is yet to be finished. The latest numbers suggest a preliminary opening date of October 2020 and estimated total costs of 6.4 Billion Euros \cite{Berlin2020}. Not explicitly suspecting any criminal activity or intended delays during the construction of the airport BER, we nevertheless want to note that the cashflow in large building projects can be complex and difficult to comprehend. Therefore, we want to suggest a solution to ease this process, increase the transparency and provide a starting point for in-depth auditioning.

\subsection{Implementation} \label{implementation}
In the following, the solution approach and the implementation of our cashflow use case is explained in detail. 

\subsubsection{Setup} \label{setup}
Our cashflow prototype was developed using Hyperledger Fabric version 1.4 and its JavaScript-based SDK. The operating system we used to set up the blockchain was \textit{MAC OS 10.15 Catalina}. 
The setup of Hyperledger Fabric requires the developer to install the following programming languages and tools with the given versions: 

\begin{itemize}
	\item \textit{Go} -- version 1.13.4 
	\item \textit{Node.js} -- version 10.17.0, \\ including \textit{NPM} -- version 6.11.3
	\item \textit{Python 2} -- version 2.7
	\item \textit{Docker} -- version 18.03.1, \\ including \textit{Docker Compose} -- version 1.21.0
	\item \textit{XCode} -- when working in MAC OS environment
	\item \textit{Homebrew} -- when working in MAC OS environment
	\item The tools \textit{curl} and \textit{wget}
\end{itemize} 
Additionally, the binaries and images needed to run our Fabric example have to be downloaded by switching to the project directory and executing \ldots
\begin{minted}[tabsize=0,breaklines,fontsize=\small]{bash}
curl -sSL http://bit.ly/2ysbOFE | bash -s. 
\end{minted}
It is important to note that Fabric is still in active development and is hence very sensitive to version changes of the tools it requires. Therefore, it is advisable to always use the versions recommended in its manual, or the ones listed in this documentation. \\
Finally, the entire code required to run our Cashflow prototype needs to be downloaded from our \textit{GitHub} repository\footnote{Official GitHub repository of our Cashflow project:\\ \url{https://github.com/lukaschoebel/cashflow}. }. 

\subsubsection{Approach} \label{approach}
The general approach we used to implement our Cashflow use case is as following: 
\begin{enumerate}
	\item Take the official \textit{fabric-samples} repository from the project's developers, and use its \textit{fabcar} example as a starting point for our own implementation. 
	\item Give the downloaded samples a test run to ensure that everything works as intended. 
	\item Implement our own logic for the concepts of user registration, chaincode transactions, and the application API. 
	\item Make sure that the prototype architecture reflects our use case, and remove unused elements of the original samples. 
\end{enumerate}

\subsubsection{Starting up Fabric} \label{starting-up-fabric}
In order to first start up the network, the \textit{startFabric.sh} script, which finds itself in our \textit{cashflow} folder, needs to be executed.
\begin{minted}[tabsize=0,breaklines,fontsize=\small]{bash}
	sh startFabric.sh
\end{minted}
The script creates a small blockchain network, comprised of several peers, an orderer node, and a certificate authority. For each of these elements it starts up an own Docker container to ensure that the blockchain components are initially isolated from each other. Additionally, the script installs a JavaScript version of the smart contract on each of the blockchain nodes. \\
The created docker containers can easily be checked by running the \mintinline{bash}{docker ps} command. It should show up the docker containers which have been set up. \\
After running the script, the node application dependencies now need to be installed in order to be able to interact with the instantiated ledger \cite{Gaski2019}. The following command installs all application dependencies defined in the $ package.json $ file \ldots
\begin{minted}[tabsize=0,breaklines,fontsize=\small]{bash}
	npm install
\end{minted}
Including the smart contract besides other dependencies, this installation is essential to make use of identities, user wallets, channel gateways as well as to submit transactions to the network. As soon as \mintinline{bash}{npm install} completes, everything is ready to go. 

\subsubsection{Enrolling admin user}
Once the blockchain network has been set up, we can now enroll our first admin user. This is done by executing the $ enrollAdmin.js $ file with 
\begin{minted}[tabsize=0,breaklines,fontsize=\small]{bash}
	node enrollAdmin.js
\end{minted}
In particular, the command creates a wallet for the admin user, which is comprised of a private and a public key, as well as a X.509 certificate \cite{Gaski2019}. 
Afterwards, the admin user will be responsible for the entire administration of the Blockchain network. 

\subsubsection{Registering users} \label{registering-users} % - script fragments and execution
While the admin user will be managing the system, we also need several ordinary users to interact with the ledger in the scope of our Cashflow use case. 
For this purpose, we create three users, who will fulfill the three different roles defined in our use case description in \ref{motivation}. In particular, we create the three users shown in Table \ref{tbl:users}.  
\begin{table}[ht]
	\begin{center}
		\addtolength{\tabcolsep}{-2pt} % make table slimmer
		\noindent\begin{tabular}{|c|c|c|} % remove tab before table
			\hline 
			\textbf{User Name} & \textbf{User Role} & \textbf{User Permissions} \\ 
			\hline 
			State Agency & Authority & queryAll \\ 
			\hline 
			Construction Company & Organizer & query, queryAll, \\ & & create, sign, change \\ 
			\hline 
			Architect & Contractor & query, create, \\ & & sign, change \\ 
			\hline 
		\end{tabular} 
	\end{center}
	\caption{Users to register with their respective roles and permissions. }
	\label{tbl:users}
	\vspace{-6mm} % reduce space after table
\end{table}
In order to register the users with their respective user roles, the following command needs to be executed: 
\begin{minted}[tabsize=0,breaklines,fontsize=\small]{bash}
	node registerUser.js -a State\ Agency -o Construction\ Company -c Architect
\end{minted}
Here, the flags $ -a $, $ -o $, and $ -c $, signify the user roles $ authority $, $ organizer $, and $ contractor $, respectively. The argument after each flag gives the name of the user with the specified user role. 

\subsubsection {Creating the smart contract} \label{smart-contract}
After registering the users, the client can interact with the ledger by using transactions, which were predefined in the smart contract. For our Cashflow use case, we created the smart contract called $ cashflow.js $, which encompasses all the functionality that is required for our proof-of-concept. \\
At the beginning of the contract, there is a function called $ initLedger $. This function is different from other transactions defined in the contract in that it gets called automatically each time the chaincode of this contract is deployed onto the Blockchain. This way, it can be conveniently used to predefine some original data stored on the ledger. 
We make use of it by creating three legal agreements between our Construction Company and some contractors, which have already been signed. \\
Defining the core functionality of the proposed network, the following transactions can be executed on the ledger \ldots
\begin{itemize}
	\item queryAgreement
	\item queryAll
	\item createAgreement
	\item signAgreement
	\item changeAgreement
\end{itemize}
Each of the transactions takes its required attributes as arguments. Also, depending on the user that issued the transaction call, it is being determined whether this user is allowed to execute this particular transaction. The access control is required to ensure that users can only execute transactions to which they are authorized, as previously explained in \ref{registering-users}. A user with the role \textit{contractor}, for instance, is not allowed to execute the \textit{queryAll} transaction to gain insight into all the legal agreements on the Blockchain. % might mention the unique ids if we need to

\subsubsection{Creating the application} \label{application}
Having created the smart contract, the only piece missing to be able to interact with the ledger is the application interface. This is implemented in the $ query.js $ JavaScript file, which can also be found in our GitHub repository. 
In its core, $ query.js $ takes as arguments the role of the user that wants to execute some transaction, the name of the transaction, and its parameters. These arguments are then extracted from the command line, and the corresponding transaction is called on the chaincode. At this point, two different variants are available to call a transaction. The first one is the function $ evaluateTransaction $, which is used for read-only requests on the ledger. The second one is $ submitTransaction $, and has to be called whenever write requests on the blockchain need to be executed. Therefore, we are choosing one of these two functions to be called on the smart contract depending on the user input. We ''evaluate'' the transactions $ query $ and $ queryAll $ and ''submit'' the transactions $ create $, $ sign $, and $ change $. 

%TODO Maybe separate section for this part with an example walkthrough like in presentation?
\subsubsection{Interacting with the blockchain}
With all the setup and implementation being done, the Blockchain system can now be interacted with by issuing transactions to it from the command line. 
Assume the following example: 

\begin{minted}[tabsize=0,breaklines,fontsize=\small]{bash}
	node query.js organizer create LAG4 52ABC1042 10M Construction\ Company Architect
	node query.js contractor sign LAG4
	node query.js contractor query LAG4
	node query.js authority queryAll
\end{minted}
In this example our organizer Construction Company creates and automatically signs a new legal agreement with the following parameters:
\begin{itemize}
	\item id: ''LAG4''
	\item hash: ''542ABC1042''
	\item cash amount: ''10M''
	\item partner\_1: ''Construction Company''
	\item partner\_2: ''Architect''
\end{itemize}	
After that, the legal agreement ''LAG4'' also gets signed by the contractor Architect, who then retrieves the agreement to check whether the signature is already there. \\
Finally, the regulation authority queries all the legal agreements deployed on the network to trace the flow of money within the construction project. 
	
\subsection{Advantages and Limitations} \label{evaluation}
% Benefits
The proposed prototype is able to make cash flow in large-scale building projects more transparent since the legal agreements and the included transactions can be traced. By introducing an increased degree of traceability, the proof-of-concept provides a first starting point for in-depth auditioning and enables to detect cash leaks or suspicious manipulations. \\
% Limitations
However, the system should be seen as a redundant alternative to the legal agreements on paper and cannot be utilized to prevent criminal activity all by itself. In addition, it needs to be noted that the suggested Fabric prototype will not work sufficiently if its environment and all included entities are entirely corrupt. 
