
\subsection{Summary}
%This work gave a brief introduction to the history and technical characteristics of Hyperledger Fabric as well as discussed its advantages and limitations. 
Fabric is a powerful state-of-the-art blockchain project, which features a modular and flexible architecture, and is suitable for various business applications. 
In this work, we have proposed a solution based on this technology to trace the cashflow in large construction projects, and described the utilized frameworks and ideas. As a proof-of-concept, the prototype sets up a Fabric network, is able to register users and make according queries to e.g. create or obtain certain legal agreements. It ensures that money spending is kept transparent and traceable by storing legal agreements between project parties on the blockchain. Further, our implementation incorporates access control which only allows to execute functions if a given user has the required access rights.
Despite Fabric being advertised as simple to build up, we encountered some difficulties during setup of the Fabric environment. Due to partially outdated documentation and incorrectly named source files, it is first necessary to fix existing bugs, and to get a deeper understanding of the Fabric infrastructure. However, in the end, we were able to successfully implement and test our cashflow prototype, which shows that Fabric can indeed be used to create and maintain blockchain-based solutions for business applications. 

