\section{Technical Properties of Hyperledger Fabric}
In order to get a deeper understanding of Hyperledger Fabric, it is essential to characterize its architecture as well as to outline its advantages and disadvantages as a permissioned, distributed ledger framework. 

\subsection{Architecture of Hyperledger Fabric}

\subsubsection{Consensus Protocols} \label{consensus}
Fabric includes different consensus protocols which allow to adjust the technology to a specific use case and trust model \cite[p. 23]{Blummer2018}. Further, the platform provides either crash-fault-tolerant (CFT) or byzantine-fault-tolerant (BFT) \cite{Castro2001} ordering. While the first one utilizes the \textit{Raft} protocol \cite{Ongaro2014}, the latter one uses the \textit{Apache Kafka} ordering service \cite{ApacheSoftwareFoundation2017}, which in its turn utilizes \textit{ZooKeeper} \cite{ApacheSoftwareFoundation} internally. By default, the byzantine-fault-tolerance consensus protocol is used for ordering \cite[p.2]{Cachin2016}. 

\subsubsection{Membership Service Provider}
New nodes are included into the network by using a "Membership Service Provider" (MSP), which is the identity management solution used in Fabric \cite[p. 31]{Matthes2019}. The MSP provides a pluggable API, which supports several different certificate authorities (CA). One of the commonly used external certificate authorities is TLS-CA. Through this CA, the identity of clients connecting to the system is verified over the HTTPS protocol. Additionally, Hyperledger Fabric provides its own certificate authority. It is called Fabric-CA, and is utilized by default in Fabric applications. With such a certificate authority, it is ensured that only the nodes which were previously authorized to connect to the network can actually do it. 

\subsubsection{Channels}
The MSP also makes sure that only authorized clients can join certain channels and are able to execute chaincode. Within Fabric, Channels are a core feature and are responsible for establishing private and secure connections between peers belonging to the same group \cite[p. 5]{Androulaki2018}. Every such channel has a ledger associated with it, of which every connected peer maintains its own copy. This ledger stores all transactions which took place on this specific channel since its creation. Which channels can be accessed by which nodes is managed by the MSP, and hence the transactions within the channel cannot be seen from outside the channel. Each of the clients can participate on multiple channels at the same time (Fig. \ref{fig:channels}), which is frequently the case in real life. For instance, it is considered common practice for one employee to participate in multiple working groups within their company.

\begin{figure}[ht]
	\centering
	\includegraphics[width=0.7\linewidth]{Channels2.pdf}
	\caption{Channels in Hyperledger Fabric}
	\label{fig:channels}
	\Description[Channel Concept]{Channels in Hyperledger Fabric}
	\vspace{-3mm} % reduce space after figure
\end{figure}

An additional benefit of the strict separation of channels is the possibility to execute transactions concurrently whenever they take place on different channels. This is a core difference to some other Blockchain systems - like e.g. Bitcoin - and provides a significant boost to performance and scalability \cite[p. 21]{Matthes2019-2}. 

\subsubsection{Ledger} 
For its ledger, Fabric utilizes an append-only distributed physical storage \cite[pp. 4]{Androulaki2018}. As in most other blockchain-based systems, the ledger records all the incoming transactions by adding blocks to its existing data chain. These blocks are connected via hash pointers to ensure data integrity and immutability, and are replicated across the system to prevent data loss. The replication happens naturally, as each of the peers involved in a channel maintains its own copy of the ledger associated with this channel. The block storage of the ledger provides multiple built-in index structures which make random access queries and transaction search particularly convenient in Fabric. The physical storage behind the ledger is provided with a local key-value store based on either Apache's \textit{CouchDB} \cite{CouchDB} or on Google's \textit{LevelDB} \cite{LevelDB}. The choice of database depends on the particular application use case, as well as on the choice of programming languages. 

\subsubsection{Node Types}
There are three different types of nodes (\textit{peers}) in Fabric: endorsing peers, committing peers, and ordering service peers \cite[p. 5]{Androulaki2018}. \\
Endorsing peers are responsible for executing proposed transactions and thus creating chaincode that can be deployed on the network. If the execution goes well, the peer sends an endorsement back to the transaction issuer. An endorsing peer is also always a committing peer. \\
Committing peers are responsible for validating the results of a transaction and for deploying (\textit{committing}) it to the network. \\
Ordering service peers are there to prevent race conditions between transactions. They order transactions according to some earlier specified algorithm, so that all transactions are committed sequentially in one batch. Ordering service nodes are only there for the ordering, i.e. they do not maintain a copy of the ledger themselves. The most commonly used ordering service utilizes \textit{Apache Kafka}, as mentioned in \ref{consensus}, which can efficiently process streams of records ordering them into blocks. 

\subsubsection{Transaction Processing}
Whenever a transaction is created on one of the channels in Fabric, it has to follow a pre-defined process in order to be included in the network. A visualization of these steps can be seen in Fig. \ref{fig:transaction-processing} and bases on \cite[p. 6]{Androulaki2018} and \cite[pp. 11]{Matthes2019-2}: 
\begin{enumerate}
	\item A node or application submits a transaction proposal to the network. 
	\item The endorser peers test-execute the chaincode involved in the transaction. 
	\item If the execution did not cause any problems, an endorsed response is returned to the transaction issuer. 
	\item The transaction issuer submits the transaction to an ordering service node. 
	\item The ordering service node orders this and other current transactions according to some ordering schema. 
	\item A batch of ordered transactions is sent to committing nodes.
	\item The committing nodes validate the transactions and commit them to the ledger. 
	\item A transaction commitment event is issued on the network \footnote{Applications, which will be introduced a bit later, can subscribe to certain events and react to them with some predefined functionality.}. 
\end{enumerate}

\begin{figure}[ht]
	\centering
	\includegraphics[width=1\linewidth]{transaction-processing.pdf}
	\caption{Transaction Processing in Fabric}
	\label{fig:transaction-processing}
	\Description[Transaction Processing]{Transaction Processing in Fabric}
	\vspace{-3mm} % reduce space after figure
\end{figure}

%TODO THIS PART CAN BE REMOVED TO CREATE MORE SPACE
In order to keep all nodes up-to-date with the current ledger state, a so-called \textit{state transfer} is performed whenever a new transaction has been committed \cite[p. 9]{Androulaki2018}. For this purpose, Fabric makes use of a gossiping protocol called \textit{Fabric gossip}, which is based on the epidemic multicast algorithm as explained in \cite{Demers1987}. The updated nodes can easily verify that the received state data is legitimate by checking the signature of the ordering service on each of the blocks. 

\subsubsection{Chaincode}
Hyperledger Fabric provides the basic framework for Blockchain applications, and container technology to host smart contracts which are called \textit{chaincode}. Within chaincode, the business logic behind the blockchain network can be defined. In its current version v. 1.6, Fabric provides chaincode SDKs for general-purpose programming languages such as Go, Node.js and Java, and offers two application SDKs for Node.js and Java. For the latter, there are also unofficially released SDKs for Python and Go \cite{HyperledgerDocs} \cite{Yamashita2019}. 

\subsubsection{Applications}
Applications based on Hyperledger Fabric are somewhat similar to Ethereum's \textit{dApps} (decentralized applications). Applications are programs that interact with the blockchain system by making calls on smart contracts \cite{Gaski2019}. As such, some applications provide a rudimentary interface, like for instance a range of console commands, to conveniently interact with the chaincode. Others include a graphical user interface (GUI), which makes them more user-friendly, and thus facilitates deployment of transactions according to some business logic. For smaller proof-of-concept applications, which aim to demonstrate the basic capabilities of Fabric, a REST API can be the interface of choice. 
%\footnote{A simple example of a REST-based application interface which interacts with Hyperledger Fabric can be seen in https://github.com/horeaporutiu/commercialPaperLoopback. The repository is managed by an official IBM Blockchain employee.}
This allows the user to interact with the ledger through a range of read and write requests.  

\subsection{Critical Discussion of Fabric}
As of date, Hyperledger Fabric maintains a prominent position among existing Blockchain projects. 
%The main reason for this is that it brings various advantages to the blockchain world and keeps advancing these in a rapid manner. 
In the following, we briefly discuss some of its strengths and weaknesses. 

\subsubsection{Advantages}
Hyperledger Fabric features a very modular architecture with its main components being fully pluggable. It supports customizable solutions for ordering, membership, endorsement, gossiping, data storage, and more \cite{Androulaki2018}. Additionally, it supports chaincode and application development in multiple programming languages and runs smart contracts in isolated \textit{Docker} \footnote{Docker is a platform providing containerization of applications. Official online resource: \url{https://www.docker.com/}, last accessed 2 Feb. 2020. } containers. All this enables simple integration into existing business applications, as well as the implementation of complex business logic in a flexible way \cite{Davies2019}. Furthermore, the physical separation of sensitive data through private channels and identity management makes sure that only those authorized may access critical data. Taking into account Fabric's better scalability and transaction throughput than most traditional Blockchain solutions, it is also more likely to be chosen over conventional distributed database systems in certain business models. \\
%In addition to that, Fabric maintains a large community of enterprise-grade partners, which support the rapid development of the project with their expertise and know-how. Also, due to the lack of a prebuilt cryptocurrency, Fabric clearly states that its main focus lies within business application development and not within digital currencies. This ensures that political controversies and state regulation are unlikely to occur in the case of Fabric. 

\subsubsection{Disadvantages and limitations}
Despite all its advantages, there are also negative aspects to mention about Hyperledger Fabric. As the project is still very new, it is prone to missing documentation, bugs, and a lack of referable examples. Due to new features being added rapidly and old ones getting deprecated, the documentation and official resources of the project cannot keep up with its current state \cite{Brock2018}. This makes it hard for developers to implement and setup the system within their own business environment. 
%We also encountered these issues during the implementation of our project, which will be introduced in chapter \ref{implementation}. 
Because of this, Fabric is currently mostly used by technology pioneers and within R\&D projects, but only rarely in production environments. \\
%Additionally, the support that Fabric receives from its numerous enterprise partners also means that it is highly influenced by these throughout its entire development cycle. One might argue that this way Fabric cannot fully enjoy the benefits of an independent open-source project. \\
Furthermore, just like the other existing blockchain systems, Fabric somewhat suffers from the lack of use cases,  where it would be provenly better than a traditional distributed DBMS \cite{Brock2018}. As the latter still offer higher transaction rates than Blockchain-based storages, it is potentially hard to compete with them despite the advantages that Fabric provides. 


