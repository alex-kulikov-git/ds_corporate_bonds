% German abstract for the CAMP report document
% Included by MAIN.TEX


\clearemptydoublepage






\vspace*{2cm}
\begin{center}
{\Large \bf Zusammenfassung}
\end{center}
\vspace{1cm}

Die vorliegende Arbeit befasst sich mit der Anpassung und Parallelisierung effizienter Skyline Algorithmen f�r moderne Hauptspeicher-Datenbanksysteme. Au�erdem richtet sie den ART Baum f�r kategorische Skyline Berechnung ein. 

Um Datenbank-Anfragen schneller bearbeiten zu k�nnen, werden einige der bekannten Skyline Algorithmen so angepasst, dass sie als Baustein direkt in eine Hauptspeicher-Datenbank aufgenommen werden k�nnen. Daraufhin werden moderne Parallelisierungstechniken vorgestellt und auf die Skyline Algorithmen angewendet. Die gemessenen Performanzergebnisse bezeugen, dass die vorgeschlagenen Parallelisierungsans�tze die Laufzeit der Algorithmen in parallelisierter Umgebung verbessern konnten. 

Dar�ber hinaus wurde der ART Baum dazu verwendet, den neuartigen Algorithmus SARTS zu entwerfen. Der Algorithmus eignet sich besonders gut zur progressiven Berechnung der Skyline in Online-Umgebungen. Im Rahmen der durchgef�hrten Tests zeigt der Algorithmus sehr solide Laufzeiten auch bei gr��eren Tupelmengen und verbraucht dabei bis zu 20-mal weniger Speicher als sein direkter Vorg�nger ST-S. %Letzteres verdankt er dem effizienten Speicherumgang des ART Baums. 

%Mit steigender Nachfrage nach interaktiven Services im Internet w�chst auch der Bedarf nach effizienten Algorithmen, welche sich mit der Verarbeitung der entstehenden Datenmengen befassen. Eine der heutzutage immer h�ufiger eingesetzten Methoden zur Filterung der Datenbasis ist die Berechnung der Skyline einer Menge von Tupeln. Zu der Skyline geh�ren diejenigen Tupel, welche interessant f�r den Anfragesteller sind. Ein Tupel gilt als interessant, sofern es von keinem anderen Tupel aus der Datenmenge dominiert wird. 


