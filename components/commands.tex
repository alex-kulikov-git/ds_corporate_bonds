% Commands to be used within the TUM report document
% Included by MAIN.TEX
% Please include your own cool commands here. 
% Be only sure to comment it sufficiently so others can use it.

%-------------------------------------------------------------
%                      Own Commands
%-------------------------------------------------------------


%-------------------------------------------------------------
% math stuff -------------------------------------------------

% nice R, N, C
\newcommand{\nat}{\mathbb{N}}
\newcommand{\real}{\mathbb{R}}
\newcommand{\compl}{\mathbb{C}}



% norm
\newcommand{\norm}[1]{\left\| #1 \right\|}

% un demi
\newcommand{\half}{\frac{1}{2}}

% parantheses
\newcommand{\parenth}[1]{ \left( #1 \right) }
\newcommand{\bracket}[1]{ \left[ #1 \right] }
\newcommand{\accolade}[1]{ \left\{ #1 \right\} }
%\newcommand{\angle}[1]{ \left\langle  #1 \right\rangle }

% partial derivative: %#1 function, #2 which variable
% simple / single line version
\newcommand{\pardevS}[2]{ \delta_{#1} f(#2) }
% fraction version
\newcommand{\pardevF}[2]{ \frac{\partial #1}{\partial #2} }

% render vectors: 3 and 4 dimensional
\newcommand{\veciii}[3]{\left[ \begin{array}[h]{c} #1 \\ #2 \\ #3	\end{array} \right]}
\newcommand{\veciv}[4]{\left[ \begin{array}[h]{c} #1 \\ #2 \\ #3 \\ #4	\end{array} \right]}

% render matrices: 3  dimensional (arguments in row first order)
\newcommand{\matiii}[9]{\left[ \begin{array}[h]{ccc} #1 & #2 & #3 \\ #4 & #5 & #6 \\ #7 & #8 & #9	\end{array} \right]}
%DOESN'T WORK,DON'T KNOW WHY \newcommand{\mativ}[16]{\left[ \begin{array}[h]{cccc} #1 & #2 & #3 & #4 \\ #5 & #6 & #7 & #8 \\ #9 & #10 & #11 & #12 \\ #13 & #14 & #15 & #16 \end{array} \right]}


%-------------------------------------------------------------
%-------------------------------------------------------------


%-------------------------------------------------------------
% some abreviations ------------------------------------------
\newcommand{\Reg}{$^{\textregistered}$}
\newcommand{\reg}{$^{\textregistered}$ }
\newcommand{\Tm}{\texttrademark}
\newcommand{\tm}{\texttrademark~}
\newcommand {\bsl} {$\backslash$}

%-------------------------------------------------------------
%-------------------------------------------------------------


%-------------------------------------------------------------
% formating --------------------------------------------------

% Theorem & Co environments and counters
\newtheorem{theorem}{Theorem}[chapter]
\newtheorem{lemma}[theorem]{Lemma}
\newtheorem{corollary}[theorem]{Corollary}
\newtheorem{remark}[theorem]{Remark}
\newtheorem{definition}[theorem]{Definition}
\newtheorem{equat}[theorem]{Equation}
\newtheorem{example}[theorem]{Example}
%\newtheorem{algorithm}[theorem]{Algorithm}

% inserting figures
\newcommand{\insertfigure}[4]{ % Filename, Caption, Label, Width percent of textwidth
	\begin{figure}[htbp]
		\begin{center}
			\includegraphics[width=#4\textwidth]{#1}
		\end{center}
		\vspace{-0.4cm}
		\caption{#2}
		\label{#3}
	\end{figure}
}




% referecing figures

\newcommand{\refFigure}[1]{ %label
	figure \ref{#1}
}
\newcommand{\refChapter}[1]{ %label
	chapter \ref{#1}
}

\newcommand{\refSection}[1]{ %label
	section \ref{#1}
}

\newcommand{\refParagraph}[1]{ %label
	paragraph \ref{#1}
}

\newcommand{\refEquation}[1]{ %label
	equation \ref{#1}
}

\newcommand{\refTable}[1]{ %label
	table \ref{#1}
}




\newcommand{\rigidTransform}[2]
{
	${}^{#2}\!\mathbf{H}_{#1}$
}

%code, in typewriter
\newcommand{\code}[1]
 {\texttt{#1}}

% comment that appears on the border - very practical !!!
\newcommand{\comment}[1]{\marginpar{\raggedright \noindent \footnotesize {\sl #1} }}

% page clearing
\newcommand{\clearemptydoublepage}{%
  \ifthenelse{\boolean{@twoside}}{\newpage{\pagestyle{empty}\cleardoublepage}}%
  {\clearpage}}


%-------------------------------------------------------------
%-------------------------------------------------------------


\newcommand{\etAl}{\emph{et al.}\mbox{ }}