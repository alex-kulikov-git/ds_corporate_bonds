\chapter{Conclusion} \label{chapter:Conclusion}
To conclude the presented work, a brief summary of the results as well as an outlook to future research will be provided. 

\section{Summary}
In the scope of the Interdisciplinary Project on the topic \textit{Leads and Lags of Corporate Bonds and Stocks}, a data extraction tool tailored to the interface of the Refinitiv Datastream product has been developed. It allows the user to conveniently enter request details for the desired financial data and to download it with significantly lesser overhead than manually via request tables. To prepare the downloaded data for future usage, a comprehensive formatting and cleaning procedure has been developed and documented. In order to make the extracted bond data comparable with equities, a matching algorithm based on the fuzzy string matching and the CUSIP-6 approaches has been derived, and can be extended to further use cases. Finally, a brief statistical analysis of the extracted data, in particular of calculated corporate bond returns has been conducted, and found to be in-line with widely accepted assumptions of the underlying statistical distributions. A basic analysis of the lead and lag relationships between corporate bond and stock returns has produced statistically significant results, according to which stock returns are faster in incorporating new information, and thus predict future corporate bond returns in the case of high-yield bonds. 

\section{Outlook}
While the bond and equity data extracted from Datastream is of relatively high quality, it could benefit greatly from a cross-check with corresponding data samples from other financial data providers, such as e.g. Bloomberg or Markit. Both for this purpose, and to further improve the existing bond-stock matching, it is recommended to estimate automated download potential for other financial data sources. A notable mention would be for example the Bloomberg Python API and its likes. For the matching procedure specifically, it would make sense to search for and evaluate additional unique company identifiers to make key-based matching globally possible. 

Concerning the statistical analysis part, I suggest to conduct a more complex and intensive series of regression analyses to extend the existing lead and lag relationship results. The returns for the regression could be prepared portfolio-wise and the model chosen to be a non-linear one to improve on the current $r^2$ statistic by reducing variance currently not induced by the independent variable. If desired to go a step further, a neural network approach with 2 to 5 hidden layers could be tried to fit the prediction model more tightly. The need for such an extensive approach should be analyzed thoroughly though, as there might be too few learnable features, which can cause a too complex model to overfit. Finally, the calculated regression model could be tested in real life for newly incoming bond data over a one or more years period to estimate its practical usability for the most reliable bond classes and regions. 