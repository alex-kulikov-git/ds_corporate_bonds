\chapter{Introduction} \label{chapter:Introduction}
In the scope of an Interdisciplinary Project (\textit{IDP} in the following) at the Technical University of Munich, the lead and lag relationship between corporate bond and stock returns had to be analyzed. With the needed stock data already provided, the first step of the IDP was to develop a tool which would be able to automatically extract static and time series data from the Thomson Reuters Datastream financial database. In the second step of the IDP, the extracted bond data had to be cleaned and prepared for further analysis by various transformation techniques. Additionally, a matching approach to join the stock and bond datasets by issuing company had to be developed. In the third and final step of the IDP, the resulting bond-stock database had to be checked for any sort of lead and/or lag relationships within bond and stock return pairs. 
 
\section{Motivation}
% Was soll diese Arbeit bewirken und warum ist sie relevant?
The main goal of this IDP is to investigate the relation between corporate bond returns and stock returns by analyzing the underlying historical and current financial data.
In the first part of this IDP, an international, extensive corporate bond database will be built up for further research. The initial data will be taken from the financial market information provider Refinitiv. To this end, some code for automated data extraction in an appropriate programming language will be developed, tested, and applied to extract the data in the most convenient manner. The code should be flexible enough to be used for data updates and/or further data extensions. As financial market data tends to be error-prone, the received information might need to be scanned for outliers and cleaned in an appropriate way before conducting further analysis based on it.
At this point, if there is some spare time in the schedule, the data received from the Refinitiv database may be compared to financial data extracted from a similar database provided by Bloomberg through a so-called Bloomberg Terminal. This step would allow for more clean and reliable corporate bond information as a starting point for the next project step.
In the second part of this IDP, I will be provided with a comprehensive data sample of international stock returns. By merging this data with the acquired corporate bond dataset, the so-called lead-lag-relation between stock and bond returns will be tested by applying suited analysis techniques. This will allow us to identify which market is faster in incorporating new information.
In addition to the data science part of the IDP, upon completion of this project, I will have acquired substantial knowledge about capital market databases, fixed income research, empirical data analysis, and the functioning of financial markets in general.
Among others, these skills are of high practical relevance for jobs in Banking, Asset Management, and Fintech.

\section{Methods}
% Gliederung der Arbeit
The present work is organized as following. At first, the necessary preliminaries with focus on the skyline operator are given. A brief overview of related work and the existing skyline algorithms takes place, with special emphasis on Naive- and Block-Nested-Loops, Divide-and-Conquer and ST-S algorithms. Hereafter, this paper briefly reviews the main advantages of the ART tree in the context of databases, and proposes a novel skyline algorithm called SARTS, which makes efficient use of the tree for dominance checks. The new algorithm and its implementation approaches are presented in greater detail. Afterwards, some of the modern parallelization techniques are first explained and then applied to the given skyline algorithms. At this point, two different query pipelining models are introduced: volcano and produce/consume. Then, an evaluation of this work's results takes place. For this purpose, various performance tests are conducted and their outcomes discussed. In the conclusion to this work, the main results and proposals of the paper are once again summarized, and an outlook to possible future work is given. 
 


