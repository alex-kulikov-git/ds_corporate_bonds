\chapter{Introduction} \label{chapter:Introduction}
In the scope of an Interdisciplinary Project (\textit{IDP} in the following) at the Technical University of Munich, the lead and lag relationship between corporate bond and stock returns has to be analyzed. With the needed stock data already provided, the first step of the IDP is to develop a tool which would be able to automatically extract static and time series data from the Thomson Reuters Datastream financial database. In the second step of the IDP, the extracted bond data has to be cleaned and prepared for further analysis by various transformation techniques. Additionally, a matching approach to join the stock and bond datasets by issuing company needs to be developed. In the third and final step of the IDP, the resulting bond-stock database has to be checked for any sort of lead and/or lag relationships within bond and stock return pairs. 

The extracted corporate bond data, both static and time series, has a wide array of applications, ranging from descriptive historical applications to predictive models, and will thus be of significant value for future financial research. In order to make use of the most recent market trends and developments, the acquired bond database, as well as the existing stock information, should be extendable, such that recent data can be accumulated in a continuous manner. This is where the tool for automated data extraction from Refinitiv Datastream comes into play. It has to provide capabilities to download financial securities data in a convenient and seamless manner, as far as technology allows. Since corporate bonds and their qualities as an investment instrument are often compared to equities, it only makes sense to additionally develop an efficient approach to analyze the two asset classes "side-by-side". At this point, a suitable matching mechanism to join bond and equity data by their issuing company is of paramount importance, and can be used in many different analysis scenarios. At the end of any financial analysis, researchers are always interested in the insights into the functioning of financial markets and the optimal investment thesis which arises therefrom. Analyzing the lead and lag relationship of corporate stocks and bonds can provide such an investment thesis based on the efficiency with which the two asset classes incorporate new information in their pricing. From a practical perspective, if I, as an investor, knew that today's stock returns predict bond returns in the next month, I could allocate my assets accordingly to gain a higher-than-usual profit from my investment. While the constraints and influence of market equilibrium conditions on such an investment strategy could be analyzed separately, the gain of such knowledge is undoubtedly of significant academical and practical importance. 


