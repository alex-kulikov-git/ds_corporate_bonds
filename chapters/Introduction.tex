\chapter{Introduction} \label{chapter:Introduction}
In the scope of an Interdisciplinary Project (\textit{IDP} in the following) at the Technical University of Munich, the lead and lag relationship between corporate bond and stock returns had to be analyzed. With the needed stock data already provided, the first step of the IDP was to develop a tool which would be able to automatically extract static and time series data from the Thomson Reuters Datastream financial database. In the second step of the IDP, the extracted bond data had to be cleaned and prepared for further analysis by various transformation techniques. Additionally, a matching approach to join the stock and bond datasets by issuing company had to be developed. In the third and final step of the IDP, the resulting bond-stock database had to be checked for any sort of lead and/or lag relationships within bond and stock return pairs. 
 
\section{Motivation}
% Was soll diese Arbeit bewirken und warum ist sie relevant?
Various online services have seen a significant rise in popularity in the last several years. The application areas range from flight booking to apartment rental. At the roots of any such service there is a massive database, containing relevant information on every item of the underlying dataset. In order to extract the database entires that are required for a particular use case, different operators can be applied to the dataset. One of the most useful filtering operators is the skyline operator. While multiple algorithms have been developed to efficiently compute the skyline of a set of tuples, there are still some optimization aspects that have not yet been covered by extensive research. 

This work in specific takes aim at reducing the computation time of some of the most prominent skyline algorithms by parallelizing them for modern CPU architectures. In addition to that, a novel skyline algorithm called SARTS is presented, which exploits the highly efficient memory usage of the ART tree \cite{art}. The algorithm was developed specifically for datasets based on categorical attributes and makes use of some modern optimization approaches in this area. 

\section{Methods}
% Gliederung der Arbeit
The present work is organized as following. At first, the necessary preliminaries with focus on the skyline operator are given. A brief overview of related work and the existing skyline algorithms takes place, with special emphasis on Naive- and Block-Nested-Loops, Divide-and-Conquer and ST-S algorithms. Hereafter, this paper briefly reviews the main advantages of the ART tree in the context of databases, and proposes a novel skyline algorithm called SARTS, which makes efficient use of the tree for dominance checks. The new algorithm and its implementation approaches are presented in greater detail. Afterwards, some of the modern parallelization techniques are first explained and then applied to the given skyline algorithms. At this point, two different query pipelining models are introduced: volcano and produce/consume. Then, an evaluation of this work's results takes place. For this purpose, various performance tests are conducted and their outcomes discussed. In the conclusion to this work, the main results and proposals of the paper are once again summarized, and an outlook to possible future work is given. 
 


